% !TEX TS-program = xelatex
% !TEX encoding = UTF-8 Unicode
% !Mode:: "TeX:UTF-8"

% ONLINE https://www.overleaf.com/project/644652bbe900dd5bbd54ddca

\documentclass{resume}
\usepackage{zh_CN-Adobefonts_external} % Simplified Chinese Support using external fonts (./fonts/zh_CN-Adobe/)
% \usepackage{NotoSansSC_external}
% \usepackage{NotoSerifCJKsc_external}
% \usepackage{zh_CN-Adobefonts_internal} % Simplified Chinese Support using system fonts
\usepackage{linespacing_fix} % disable extra space before next section
\usepackage{cite}

\begin{document}
\pagenumbering{gobble} % suppress displaying page number

\name{酒嘉年}

\basicInfo{
  \email{jojiu7609@outlook.com} \textperiodcentered\ 
  \phone{(+86) 166-3892-3406} \textperiodcentered\ 
  \github[Ninglo]{https://github.com/Ninglo}}

\section{\faGraduationCap\  教育背景}
\datedsubsection{\textbf{中国药科大学}, 南京}{2018.09 - 2022.06}
\textit{学士}\ 药物分析

\section{\faUsers\ 工作经历}
\datedsubsection{\textbf{字节跳动} \space 前端开发 (CloudIDE 方向)}{2021.01 - 至今}
% \role{TypeScript, React, nodejs}{前端开发 (CloudIDE 方向)}

%region CloudIDE 前端场景解决方案产出
\datedsubsection{CloudDev 项目}{}
\role{TypeScript, React, Go, Java}{插件侧研发负责人}

\begin{onehalfspacing}
CloudDev 项目插件侧架构设计
\begin{itemize}
  \item 能力层和视图层分层: 由于不同 IDE 间的交互逻辑差异极大, 同时插件开发语言也不同, 所以在视图层实现跨插件的能力复用基本不可能. 但是能力层实现逻辑类似, 因此可以将插件分为能力层和视图层两个部分, 使得能力层的复用成为可能
  \item \(Dev Server Protocol\): 为了实现底层能力在双端插件的复用,基于经验抽象出 DSP 协议,由 CLI 实现底层能力并通过 DSP 向插件侧输出
\end{itemize}
\end{onehalfspacing}

\begin{onehalfspacing}
项目工程化建设
\begin{itemize}
  \item 构建发布: 插件侧使用 esbuild 作为构建工具、webview 部分使用 vite 构建配置项目构建发布自动化流程,完成合码后即可全自动构建发布
  \item 调试能力: 基于 vscode 提供的各配置内容,实现项目的一键启动调试能力,降低新人上手成本
  \item monorepo: 使用 rushjs 管理 js 侧项目 monorepo
\end{itemize}
\end{onehalfspacing}

\begin{onehalfspacing}
完成 CloudDev 项目 V1 版本开发
\end{onehalfspacing}
%endregion

%region CloudIDE 前端场景解决方案产出
\datedsubsection{CloudIDE 前端场景解决方案产出}{}
% \role{产品方案}{}
\begin{onehalfspacing}
通过梳理前端研发在开发阶段的全部流程,分析 CloudIDE 可以介入提升用户研发效率和体验的部分,从而产出面向前端场景的 CloudIDE 解决方案
\begin{itemize}
  \item 云端代理: 基于公司内的流量分发体系, 将用户调试启动的 dev-server 注册至线上 page-server 的泳道环境中, 方便用户的自测调试或提供给 QA\&PM 预览
  \item MR\&CR in IDE: 将 MR\&CR 能力使用插件实现, 使用户在 IDE 中更为沉浸式的完成开发完整流程, 确立 IDE 作为开发阶段核心入口的地位
  \item 预构建: 支持用户配置自定义安装构建脚本, 在用户创建环境前完成下载依赖、构建项目的行为, 保证用户创建 CloudIDE 后可以极速进行开发
\end{itemize}
\end{onehalfspacing}
%endregion

%region VSCode 项目
\datedsubsection{VSCode 项目}{}
\role{Typescript, Node.js}{项目核心开发成员}
\begin{onehalfspacing}
% 优雅的 \LaTeX\ 简历模板, https://github.com/billryan/resume
\begin{itemize}
  \item VSCode 能力加强:基于公司内场需求,对 VSCode 做定制化能力的开发
  \item 补全 VSCode 并未开源的周边生态能力(配置同步后端服务、插件市场等)
  \item 适配公司内部基建,确保 vscode-server 在公司内部环境内正常的运行和使用
\end{itemize}
\end{onehalfspacing}
%endregion

%region 其他工作内容
\datedsubsection{其他工作内容}{}
% \role{Typescript, Node.js}{项目核心开发成员}
\begin{onehalfspacing}
% 优雅的 \LaTeX\ 简历模板, https://github.com/billryan/resume
\begin{itemize}
  \item CloudIDE 环境控制台、环境启动中转页页面开发 \(React\)
  \item 补全 VSCode 并未开源的周边生态能力(配置同步后端服务、插件市场等)\(nodejs\)
  \item Cloud Workspace 插件开发,支持用户使用 native VSCode 连接 CloudIDE 容器 \(nodejs\)
\end{itemize}
\end{onehalfspacing}
%endregion

% Reference Test
%\datedsubsection{\textbf{Paper Title\cite{zaharia2012resilient}}}{May. 2015}
%An xxx optimized for xxx\cite{verma2015large}
%\begin{itemize}
%  \item main contribution
%\end{itemize}

\section{\faCogs\ IT 技能}
% increase linespacing [parsep=0.5ex]
\begin{itemize}[parsep=0.5ex]
  \item 熟练掌握HTML5、CSS3、JavaScript
  \item 熟练掌握React库,熟悉hooks、mobx等React生态中的状态管理方案
  \item 熟悉Node.js和Go语言,有生产环境项目落地经验
\end{itemize}

% \section{\faHeartO\ 获奖情况}
% \datedline{\textit{第一名}, xxx 比赛}{2013 年6 月}
% \datedline{其他奖项}{2015}

\section{\faInfo\ 其他}
% increase linespacing [parsep=0.5ex]
\begin{itemize}[parsep=0.5ex]
  \item VSCode Contributor: https://github.com/microsoft/vscode/pull/160863
\end{itemize}

%% Reference
%\newpage
%\bibliographystyle{IEEETran}
%\bibliography{mycite}
\end{document}
